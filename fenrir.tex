\documentclass{jsarticle}
% \documentclass[b4paper,landscape,14pt]{jsarticle}
\title{}
\author{}
\date{
% \number\year 年 \number\month 月
}
\usepackage{fenrir_v1_4_0}
\usepackage{ethm_v1_1_0}
\mathtoolsset{showonlyrefs=true}

\begin{document}
% \maketitle
\setcounter{section}{4}
\section{Stochastic Integration}
\setcounter{subsection}{4}
\subsection{Girsanov's Theorem}

\setcounter{thm}{22}
\begin{shadebox}
    \begin{thm}
    \label{thm:523}
        $L$: CLM s.t. $L_0=0.$
        以下の性質を考える:
        \begin{enumerate}[label=(\roman*)]
            \item
            $E[\exp\frac{1}{2}\gen{L,L}_{\infty}]<\infty$
            (Novikov's criterion\footnote{criterion(クライテリオン):判定条件})
            \item
            $L$: UIMかつ $E[\exp\frac{1}{2}L_{\infty}]<\infty$
            (Kazamaki's criterion\footnote{風巻紀彦(\url{https://researchmap.jp/read0009617})})
            \item
            $\mathcal{E}(L)$: UIM
        \end{enumerate}
        このとき,(i) $\Rightarrow$ (ii) $\Rightarrow$ (iii) の順で成立.
    \end{thm}
\end{shadebox}


\begin{proof}
    \begin{description}
        \item[(i) $\Rightarrow$ (ii)]
        $\gen{L,L}_\infty<\exp\frac{1}{2}\gen{L,L}_\infty$ と(i) より $E[\gen{L,L}_{\infty}]\le E[\exp\frac{1}{2}\gen{L,L}_{\infty}]<\infty$ であり,さらにThm.4.13より $L$ は $L^2$-bdd.な連続martingale.
        \begin{screen}
            $\because)$ ($\gen{L,L}_\infty<\exp\frac{1}{2}\gen{L,L}_\infty$ であること)
            $g(x)\coloneqq e^{x/2}-x$ と定めると $g(x)\ge g(2\log 2)=2(1-\log 2)>0.$
        \end{screen}
        
        つまり $L^1$-bdd.な連続martingaleなのでProp.3.19より $\Exists L_\infty=\lim_{t\to\infty}L_t$ a.s.
        よってThm.3.21より $L$: UIM.
        このとき
        \begin{align}
            \exp\frac{1}{2}L_{\infty}
            = (\exp L_\infty)^{1/2}
            &= \left(\exp\left(L_\infty-\frac{1}{2}\gen{L,L}_{\infty}+\frac{1}{2}\gen{L,L}_{\infty}\right)\right)^{1/2} \\
            &= \left(\exp\left(L_\infty-\frac{1}{2}\gen{L,L}_{\infty}\right)\right)^{1/2}
            \left(\exp\frac{1}{2}\gen{L,L}_{\infty}\right)^{1/2} \\
            &= (\mathcal{E}(L)_{\infty})^{1/2}\left(\exp\frac{1}{2}\gen{L,L}_{\infty}\right)^{1/2}
        \end{align}
        が成り立つので
        \begin{align}
            E\left[\exp\frac{1}{2}L_{\infty}\right]
            &= E\left[(\mathcal{E}(L)_{\infty})^{1/2}\left(\exp\frac{1}{2}\gen{L,L}_{\infty}\right)^{1/2}\right] \\
            &\stackrel{\text{C-S}}{\le} (\UB{E[\mathcal{E}(L)_{\infty}]}{\le E[\mathcal{E}(L)_0]\le1})^{1/2}
            \left(E\left[\exp\frac{1}{2}\gen{L,L}_{\infty}\right]\right)^{1/2} \\
            &\le \left(E\left[\exp\frac{1}{2}\gen{L,L}_{\infty}\right]\right)^{1/2}
            < \infty.
        \end{align}
        
        \item[(ii) $\Rightarrow$ (iii)]
        $L$ がUIMであることからThm.3.22より任意のstopping time $T$ に対し $L_T=E[L_{\infty}\mid\clf_T].$
        $e^{x/2}$: convexであることからJensenの不等式より
        \begin{align}
            \exp\frac{1}{2}L_T
            = \exp\frac{1}{2}E[L_{\infty}\mid\clf_T]
            \le E\left[\exp\frac{1}{2}L_{\infty}\mid\clf_T\right].
        \end{align}

        $E[\UB{\exp\frac{1}{2}L_{\infty}}{\text{あるr.v.}}]<\infty$ という仮定より,任意のstopping time $T$ に対し $E[\exp\frac{1}{2}L_{\infty}\mid\clf_T]$ の形をしたすべてのr.v.の族は一様可積分(by Thm.3.21の証明の最初).
        さらに先ほどの評価より,任意のstopping time $T$ に対し $\exp\frac{1}{2}L_T$ の形をしたすべてのr.v.の族もまた一様可積分.

        $0<\Forall a<1$ に対し,$Z_t^{(a)}\coloneqq\exp(\frac{aL_t}{1+a})$ と定める.
        このとき
        \begin{align}
            \mathcal{E}(aL)_t
            &= \exp\left(aL_t-\frac{1}{2}\gen{aL,aL}_t\right) \\
            &= \exp\left(a^2 L_t-\frac{1}{2}\gen{aL,aL}_t+aL_t-a^2 L_t\right) \\
            &= \exp\left(a^2 L_t-\frac{a^2}{2}\gen{L,L}_t\right)\cdot\exp((1-a)aL_t) \\
            &= \left(\exp\left(L_t-\frac{1}{2}\gen{L,L}_t\right)\right)^{a^2}\left(\exp\left(\frac{aL_t}{1+a}\right)\right)^{1-a^2} \\
            &= (\mathcal{E}(L)_t)^{a^2}(Z_t^{(a)})^{1-a^2}
        \end{align}
        となる.
        よって $\Gamma\in\clf,\ T$: stopping timeに対し
        \begin{align}
            E[\bm{1}_\Gamma\mathcal{E}(aL)_T]
            &= E[\bm{1}_\Gamma(\mathcal{E}(L)_T)^{a^2}(Z_T^{(a)})^{1-a^2}] \\
            &\stackrel{\text{H\"{o}lder}}{\le} E[\mathcal{E}(L)_T]^{a^2}
            E[\bm{1}_\Gamma Z_T^{(a)}]^{1-a^2} \\
            &\le E[\bm{1}_\Gamma Z_T^{(a)}]^{1-a^2} \\
            &= E\left[\bm{1}_\Gamma\exp\left(\frac{aL_T}{1+a}\right)\right]^{1-a^2} \\
            &\stackrel{(\ast)}{\le} E\left[\bm{1}_\Gamma\exp\frac{1}{2}L_T\right]^{2a(1-a)}.
        \end{align}

        2つ目の不等号では $E[\mathcal{E}(L)_T]\le(E[\mathcal{E}(L)_0]=)1$ という性質を用いた($\mathcal{E}(L)$: 非負supermartingaleかつ $\mathcal{E}(L)_0=1$ よりProp.3.25が適用できる).
        3つ目の不等号では $\frac{1+a}{2a}>1$ に注意してJensenの不等式を用いた.
        \begin{screen}
        $\because)$ ($(\ast)$ が成り立つこと)
        $x^{\frac{1+a}{2a}}$: convexより
        \begin{align}
            E\left[\bm{1}_\Gamma\exp\left(\frac{aL_T}{1+a}\right)\right]^{\frac{1+a}{2a}}
            \le E\left[\bm{1}_\Gamma\left(\exp\left(\frac{aL_T}{1+a}\right)\right)^{\frac{1+a}{2a}}\right]
            = E\left[\bm{1}_\Gamma\exp\frac{1}{2}L_T\right].
        \end{align}

        両辺を $2a(1-a)$ 乗すれば目的の不等式を得る.
        \end{screen}
        
        任意のstopping time $T$ に対し $\exp\frac{1}{2}L_T$ の形をしたすべてのr.v.の族が一様可積分であることより,先ほどの表示から任意のstopping time $T$ に対し $\mathcal{E}(aL)_T$ の形をしたすべてのr.v.の族もまた一様可積分であることが言える.
        \begin{screen}
            $\because)$
            $\lambda>0$: fix

            $\exp\frac{1}{2}L_T$ は一様可積分なので,定義より
            \begin{align}
                \lim_{\lambda\to\infty}\sup_T E\left[\bm{1}_{\{\exp\frac{1}{2}L_T>\lambda\}}\exp\frac{1}{2}L_T\right]=0.
            \end{align}

            $\Gamma=\{\mathcal{E}(aL)_T>\lambda\}\in\clf$ に対し
            \begin{align}
                \mathcal{E}(aL)_T
                = \exp{\left(aL_T-\frac{1}{2}\gen{aL,aL}_T\right)}
                \le \left(\exp\frac{1}{2}L_T\right)^{2a}
            \end{align}
            より $\Gamma\subset \{\exp\frac{1}{2}L_T>\lambda^{1/2a}\}$ が成り立つので
            \begin{align}
                \sup_T E\left[\bm{1}_{\Gamma}\mathcal{E}(aL)_T\right]
                &\le \sup_T E\left[\bm{1}_{\Gamma}\exp\frac{1}{2}L_T\right]^{2a(1-a)} \\
                &\le \sup_T E\left[\bm{1}_{\{\exp\frac{1}{2}L_T>\lambda^{1/2a}\}}\exp\frac{1}{2}L_T\right]^{2a(1-a)}
                \longrightarrow 0\quad(\lambda\to0).
            \end{align}

            ゆえに $\mathcal{E}(aL)_T$ は一様可積分.
        \end{screen}
        
        CLMの定義より $\Forall n$ に対し $\mathcal{E}(aL)_{t\wedge T_n}$: UIMとなるstopping timeの増加列 $T_n\uparrow\infty$ が存在する.
        $0\le s\le t$ に対し $E[\mathcal{E}(aL)_{t\wedge T_n}\mid\clf_s]=\mathcal{E}(aL)_{s\wedge T_n}$,つまり $\Forall A\in\clf_s$ に対し $E[\mathcal{E}(aL)_{t\wedge T_n};A]=E[\mathcal{E}(aL)_{s\wedge T_n};A]$ が成り立ち,一様可積分性よりVitaliの収束定理が適用できるので,両辺について $n\to\infty$ とすると $\mathcal{E}(aL)$ はUIM($\implies E[\mathcal{E}(aL)_{\infty}]=1$).
        これより再びJensenの不等式を用いることで
        \begin{align}
            1=E[\mathcal{E}(aL)_{\infty}]
            \le E[\mathcal{E}(L)_\infty]^{a^2}
            E[Z_{\infty}^{(a)}]^{1-a^2}
            \le E[\mathcal{E}(L)_\infty]^{a^2}
            E\left[\exp\frac{1}{2}L_\infty\right]^{2a(1-a)}
        \end{align}
        がわかる.
        $a\to1$ とすると $E[\mathcal{E}(L)_\infty]\ge1$ より $E[\mathcal{E}(L)_\infty]=1.$
    \end{description}
\end{proof}


\subsection{A Few Applications of Girsanov's Theorem}

\textbf{Constructing solutions of stochastic differential equations(確率微分方程式の解の構成)}
$b$:有界可測関数 on $\real_+\times\real$ とする.
$\Forall (t,x)\in\real_+\times\real$ に対し $\abs{b(t,x)}\le g(t)$ を満たす関数 $g\in L^2(\real_+,\clb(\real_+),dt)$ が存在すると仮定.
これは特に
\begin{align}
    \abs{b(t,x)}=
    \begin{cases}
        \text{bdd.} &\text{on}\quad [0,A]\times\real \\
        0 &\text{on}\quad (A,\infty)\times\real
    \end{cases}
\end{align}
を満たす $A>0$ が存在するときに成り立つ($g(t):=\sup_{x\in\real}\abs{b(t,x)}$ と定めればよい)\footnote{この係数 $b$ を\textbf{ドリフト係数}という(progressiveであることも仮定したほうが良い?).}.

$B$: $(\clf_t)$-BMとする.
以下のCLM
\begin{align}
    L_t=\int_0^t b(s,B_s)dB_s
\end{align}
と関連するexponential martingale
\begin{align}
    D_t
    = \mathcal{E}(L)_t
    = \exp\left(\int_0^t b(s,B_s)dB_s-\frac{1}{2}\int_0^t b(s,B_s)^2 ds\right)
\end{align}
について考える.
$b$ の仮定よりThm.\ref{thm:523}の(i)が成り立つことは保証されるので $D$: UIM.

\begin{screen}
    $\because)$
    \begin{align}
        \exp\frac{1}{2}\gen{L,L}_\infty
        = \exp\left(\frac{1}{2}\int_0^\infty b(s,B_s)^2 ds\right)
        \le \exp\left(\frac{1}{2}\int_0^\infty g(s)^2 ds\right)
        < \infty.
    \end{align}
\end{screen}

$Q\coloneqq D_\infty\cdot P$, i.e. $A\in\clf_\infty$ に対し $Q(A)\coloneqq\int_A D_\infty dP$ と定める.
Girsanovの定理とConsequencesの(c)より
\begin{align}
    \beta_t
    \coloneqq B_t-\gen{B,L}_t
    = B_t-\int_0^t b(s,B_s)ds
\end{align}
で定まるprocessは $(\clf_t)$-BM under $Q.$

この性質はprocess $X=B$ が確率微分方程式
\begin{align}
    dX_t=d\beta_t+b(t,X_t)dt
\end{align}
の解であるような $(\clf_t)$-BM $\beta$ が確率測度 $Q$ の下で存在する,と言い換えることができる.
(この式は後述のChapter 8で検討されるタイプのものであるが,Chapter 8での記述とは異なり,ここでは関数 $b$ に正則性は仮定していない.
Girsanovの定理によりドリフト係数の正則性を仮定することなく確率微分方程式の解を構成することができる.)

\textbf{The Cameron-Martin formula}
先ほどの議論を $b(t,x)$ が $x$ に依存しない場合に限って議論する.
$g\in L^2(\real_+,\clb(\real_+),dt)$ に対し $g(t)\coloneqq b(t,x)$ と定め,また $\Forall t\ge0$ に対し
\begin{align}
    h(t)=\int_{0}^{t}g(s)ds
\end{align}
と定める.
\begin{align}
    \mathcal{H}
    = \set{h}{h(t)=\int_{0}^{t}g(s)ds\text{ for }\Forall t\ge0}
\end{align}
を\textbf{Cameron-Martin空間}という.
$h\in\mathcal{H}$ のとき,しばしば $\dot{h}=g$ と書く($g$ は超関数の意味で $h$ の導関数である).

先ほどの議論の特別な場合として,以下に定める確率測度
\begin{align}
    Q
    \coloneqq D_\infty\cdot P
    = \exp\left(\int_0^\infty g(s)dB_s-\frac{1}{2}\int_0^\infty g(s)^2 ds\right)\cdot P
\end{align}
の下で,process $\beta_t\coloneqq B_t-h(t)$ はBM.
ゆえに任意の非負可測関数 $\Phi$ on $C(\real_+,\real)$ に対し
\begin{align}
    E_P[D_\infty\Phi((B_t)_{t\ge0})]
    = E_Q[\Phi((B_t)_{t\ge0})]
    = E_Q[\Phi((\UB{\beta_t}{\text{BM under }Q}+h(t))_{t\ge0})]
    \beq{\text{同分布}} E_P[\Phi((\UB{B_t}{\text{BM under }P}+h(t))_{t\ge0})].
\end{align}

この式の最左辺と最右辺を見た
\begin{align}
    \int \Phi(B+h)dP
    = \int\exp\left(\int_0^\infty g(s)dB_s-\frac{1}{2}\int_0^\infty g(s)^2 ds\right)\Phi(B)dP
\end{align}
を\textbf{Cameron-Martinの公式}という.
次の命題では,この公式をWiener空間上のBMのcanonical construction(Section 2.2末尾参照)の特別な場合について書く.

\begin{shadebox}
    \begin{prop}[Cameron-Martinの公式]
    \label{thm:524}
        $W(dw)$: Wiener測度 on $C(\real_+,\real), h\in\mathcal{H}$

        このとき $\Forall \Phi$:非負可測関数 on $C(\real_+,\real)$ に対し
        \begin{align}
            \int \Phi(w+h)W(dw)
            = \int\exp\left(\int_0^\infty \dot{h}(s)dw(s)-\frac{1}{2}\int_0^\infty \dot{h}(s)^2 ds\right)\Phi(w)W(dw).
        \end{align}
    \end{prop}  
\end{shadebox}

\begin{remark*}
    積分 $\int_0^\infty\dot{h}(s)dw(s)$ は $W(dw)$ の下でのBM $w(s)$ について確率積分であるが,関数 $\dot{h}(s)$ は決定論的なのでWiener積分とみなすこともできる.
    よって,実はCameron-Martinの公式は確率積分やGirsanovの定理を用いないGauss系の性質から導くことができる.
    しかしこの公式をGirsanovの定理を用いて導出することは教育的である.

    Cameron-Martinの公式より,Wiener測度はCameron-Martin空間に属する関数 $h$ による平行移動の下で「quasi-invariance(準不変性)」を持つことがわかる.
    つまり(平行移動させる)写像 $w\mapsto w+h$ の下でのWiener測度 $W(dw)$ の像測度は $W(dw)$ に関する密度を持ち(絶対連続),この密度はmartingale $\int_0^t \dot{h}(s)dw(s)$ に関連するexponential martingaleの $t=\infty$ での値,つまり
    \begin{align}
        \mathcal{E}\left(\int_0^{\cdot}\dot{h}(s)dw(s)\right)_\infty
        = \exp\left(\int_0^\infty \dot{h}(s)dw(s)-\frac{1}{2}\int_0^\infty \dot{h}(s)^2 ds\right)
    \end{align}
    である.
\end{remark*}
\hrulefill

% 時間が余ったとき用
\textbf{Law of hitting times for Brownian motion with drift}~
$B$:0スタートのreal BM, $T_a\coloneqq\inf\set{t\ge0}{B_t=a}$ for $\Forall a>0$ (first hitting time)

$c\in\real$ のとき,stopping time
\begin{align}
    U_a\coloneqq\inf\set{t\ge0}{B_t+ct=a}
\end{align}
の分布を求める.

\begin{proof}
    $c=0$ のときは $U_a=T_a$ であり,Cor.2.22より $a^2/B_1^2$ と同じ分布をもつ.
    Girsanovの定理(またはCameron-Martinの公式)より,$c=0$ の特別な場合から $c$ が任意の実数の場合を導くことができる.
    
    $t>0$ を固定する.
    \begin{align}
        \dot{h}(s)=c\bm{1}_{\{s\le t\}},\quad
        h(s)=c(s\wedge t)
    \end{align}
    と
    \begin{align}
        \Phi(w)=\bm{1}_{\{\max_{s\in[0,t]}w(s)\ge a\}}\text{ for }\Forall w\in C(\real_+,\real)
    \end{align}
    に対し
    \begin{align}
        P(U_a\le t)
        &= P(\max_{s\in[0,t]}(B_s+cs)\ge a) \\
        &= P(\max_{s\in[0,t]}(B_s+c(s\wedge t))\ge a) \\
        &= P(\max_{s\in[0,t]}(B_s+h(s))\ge a) \\
        &= E[\bm{1}_{\{\max_{s\in[0,t]}(B_s+h(s))\ge a\}}]
        = E[\Phi(B+h)] \\
        &\beq{\text{C-M}} E\left[\Phi(B)\exp\left(\int_0^\infty \dot{h}(s)dB_s-\frac{1}{2}\int_0^\infty \dot{h}(s)^2 ds\right)\right] \\
        &= E\left[\bm{1}_{\{T_a\le t\}}\exp\left(c\int_0^\infty \bm{1}_{\{s\le t\}}dB_s-\frac{c^2}{2}\int_0^\infty \bm{1}_{\{s\le t\}} ds\right)\right] \\
        &= E\left[\bm{1}_{\{T_a\le t\}}\exp\left(cB_t-\frac{c^2}{2}t\right)\right] \\
        &\beq{(\ast)} E\left[\bm{1}_{\{T_a\le t\}}\exp\left(cB_{t\wedge T_a}-\frac{c^2}{2}(t\wedge T_a)\right)\right] \\
        &= E\left[\bm{1}_{\{T_a\le t\}}\exp\left(ca-\frac{c^2}{2}T_a\right)\right] \\
        &= \int_{-\infty}^{\infty} \bm{1}_{\{s\le t\}}e^{ca-\frac{c^2}{2}s}\cdot\frac{a}{\sqrt{2\pi s^3}}e^{-\frac{a^2}{2s}}\bm{1}_{\{s>0\}}ds \\
        &= \int_0^t \frac{a}{\sqrt{2\pi s^3}}e^{-\frac{a^2}{2s}}e^{ca-\frac{c^2}{2}s}ds \\
        &= \int_0^t \frac{a}{\sqrt{2\pi s^3}}e^{-\frac{1}{2s}(a-cs)^2}ds.
    \end{align}
    
    \begin{screen}
        $\because)$(($\ast$)が成り立つこと)
        $\exp\left(cB_t-\frac{c^2}{2}t\right)$: martingaleであることからCor.3.23より
        \begin{align}
            E\left[\exp\left(cB_t-\frac{c^2}{2}t\right)\mid\clf_{t\wedge T_a}\right]
            = \exp\left(cB_{t\wedge T_a}-\frac{c^2}{2}(t\wedge T_a)\right).
        \end{align}
        % さらにCor.2.22で与えられた $T_a$ の密度 $f(t)=\frac{a}{\sqrt{2\pi t^3}}e^{-\frac{a^2}{2t}}$ を用いる.
    \end{screen}
    
    この計算によりr.v. $U_a$ は $\real_+$ 上に密度
    \begin{align}
        \psi(s)
        = \frac{a}{\sqrt{2\pi s^3}}e^{-\frac{1}{2s}(a-cs)^2}
    \end{align}
    を持つことがわかる.
\end{proof}

この密度を積分することで
\begin{align}
    P(U_a<\infty)=
    \begin{cases}
        1 &\text{if } c\ge0, \\
        e^{2ca} &\text{if } c\le0
    \end{cases}
\end{align}
がわかるが,この結果は連続martingale $\exp{(-2c(B_t+ct))}$ に対しoptional stopping theoremを適用することで,より簡単に得ることができる.
\begin{screen}
    $\because)$
    $c\in\real$ に対しhitting timeのラプラス変換は
    \begin{align}
        E\left[\exp\left(-\frac{c^2}{2}T_a\right)\right]
        = e^{-\abs{c}a}
    \end{align}
    となる.
    ゆえに
    \begin{align}
        P(U_a\le t)=E\left[\bm{1}_{\{T_a\le t\}}\exp\left(ca-\frac{c^2}{2}T_a\right)\right]
    \end{align}
    の両辺について $t\to\infty$ とすると
    \begin{align}
        P(U_a<\infty)
        = e^{ca} E\left[\exp\left(-\frac{c^2}{2}T_a\right)\right]
        = e^{ca}\cdot e^{-\abs{c}a} =
        \begin{cases}
            1 &\text{if } c\ge0, \\
            e^{2ca} &\text{if } c\le0.
        \end{cases}
    \end{align}
\end{screen}
\end{document}
