\documentclass{jsarticle}
% \documentclass[b4paper,landscape,14pt]{jsarticle}
\title{}
\author{}
\date{
% \number\year 年 \number\month 月
}
\usepackage{fenrir_v1_4_0}
\usepackage{ethm_v1_1_0}
\mathtoolsset{showonlyrefs=true}

\begin{document}
% \maketitle
\setcounter{section}{4}
\section{Stochastic Integration}
\setcounter{subsection}{4}
\subsection{Girsanov's Theorem}

\setcounter{thm}{22}
\begin{screen}
    \begin{thm}
    \label{thm:523}
        $L$: CLM s.t. $L_0=0.$
        以下の性質を考える:
        \begin{enumerate}[label=(\roman*)]
            \item
            $E[\exp\frac{1}{2}\gen{L,L}_{\infty}]<\infty$
            (Novikov's criterion\footnote{criterion(クライテリオン):基準})
            \item
            $L$: CLMかつ $E[\exp\frac{1}{2}L_{\infty}]<\infty$
            (Kazamaki's criterion)
            \item
            $\mathcal{E}(L)$: UIM
        \end{enumerate}
        このとき,(i) $\Rightarrow$ (ii) $\Rightarrow$ (iii) の順で成立.
    \end{thm}
\end{screen}


\begin{proof}
    \begin{description}
        \item[(i) $\Rightarrow$ (ii)]
        (i) より $E[\gen{L,L}_{\infty}]<\infty$ であり,さらにThm. 4.13より $L$ は $L^2$-bdd. な連続martingale.
        このとき
        \begin{align}
            \exp\frac{1}{2}L_{\infty}
            = (\mathcal{E}(L)_{\infty})^{1/2}(\exp\frac{1}{2}\gen{L,L}_{\infty})^{1/2}
        \end{align}
        が成り立つので,Cauchy-Schwarzの不等式より
        \begin{align}
            E[\exp\frac{1}{2}L_{\infty}]
            &\le (E[\mathcal{E}(L)_{\infty}])^{1/2}
            (E[\exp\frac{1}{2}\gen{L,L}_{\infty}])^{1/2} \\
            &\le (E[\exp\frac{1}{2}\gen{L,L}_{\infty}])^{1/2}
            < \infty.
        \end{align}
        
        \item[(ii) $\Rightarrow$ (iii)]
        $L$ がUIMであることからThm. 3.22より任意のstopping time $T$ に対し $L_T=E[L_{\infty}\mid\clf_T].$
        Jensenの不等式より
        \begin{align}
            \exp\frac{1}{2}L_T
            \le E[\exp\frac{1}{2}L_{\infty}\mid\clf_T].
        \end{align}

        $E[\exp\frac{1}{2}L_{\infty}]<\infty$ という仮定より,任意のstopping time $T$ に対し $E[\exp\frac{1}{2}L_{\infty}\mid\clf_T]$ の形をしたすべてのr.v.の族は一様可積分.
        さらに先ほどの有界性より,任意のstopping time $T$ に対し $\exp\frac{1}{2}L_T$ の形をしたすべてのr.v.の族もまた一様可積分.

        $0<\Forall a<1$ に対し,$Z_t^{(a)}:=\exp(\frac{aL_t}{1+a})$ と定める.
        このとき
        \begin{align}
            \mathcal{E}(aL)_t
            = (\mathcal{E}(L)_t)^{a^2}(Z_t^{(a)})^{1-a^2}
        \end{align}
        となることがすぐにわかる.
        もし $\Gamma\in\clf$ かつ $T$: stopping timeならば,H\"{o}lderの不等式より
        \begin{align}
            E[\bm{1}_\Gamma\mathcal{E}(aL)_T]
            \le E[\mathcal{E}(L)_T]^{a^2}
            E[\bm{1}_\Gamma Z_T^{(a)}]^{1-a^2}
            \le E[\bm{1}_\Gamma Z_T^{(a)}]^{1-a^2}
            \le E[\bm{1}_\Gamma\exp\frac{1}{2}L_T]^{2a(1-a)}.
        \end{align}

        2つ目の不等号では $E[\mathcal{E}(L)_T]\le1$ という性質を用いた($\mathcal{E}(L)$: 非負supermartingaleかつ $\mathcal{E}(L)_0=1$ よりProp. 3.25が適用できる).
        3つ目の不等号では $\frac{1+a}{2a}>1$ に注意してJensenの不等式を用いた.
        任意のstopping time $T$ に対し $\exp\frac{1}{2}L_T$ の形をしたすべてのr.v.の族が一様可積分であることより,先ほどの表示から任意のstopping time $T$ に対し $\mathcal{E}(aL)_T$ の形をしたすべてのr.v.の族もまた一様可積分であることが言える.
        CLMの定義より $\Forall n$ に対し $\mathcal{E}(aL)_{t\wedge T_n}$: martingaleとなるstopping timeの増加列 $T_n\uparrow\infty$ が存在する.
        $0\le s\le t$ に対し $E[\mathcal{E}(aL)_{t\wedge T_n}\mid\clf_s]=\mathcal{E}(aL)_{s\wedge T_n}$ が成り立ち,一様可積分性より両辺について $n\to\infty$ とすると $\mathcal{E}(aL)$ は一様可積分.
        これより再びJensenの不等式を用いることで
        \begin{align}
            1=E[\mathcal{E}(aL)_{\infty}]
            \le E[\mathcal{E}(L)_\infty]^{a^2}
            E[Z_{\infty}^{(a)}]^{1-a^2}
            \le E[\mathcal{E}(L)_\infty]^{a^2}
            E[\exp\frac{1}{2}L_\infty]^{2a(1-a)}
        \end{align}
        がわかる.
        $a\to1$ とすると $E[\mathcal{E}(L)_\infty]\ge1$ より $E[\mathcal{E}(L)_\infty]=1.$
    \end{description}
\end{proof}


\subsection{A Few Applications of Girsanov's Theorem}

\textbf{Constructing solutions of stochastic differential equations(確率微分方程式の解の構成)}
$b$: bdd. m'ble ft. on $\real_+\times\real$ とする.
$\Forall (t,x)\in\real_+\times\real$ に対し $\abs{b(t,x)}\le g(t)$ を満たす関数 $g\in L^2(\real_+,\clb(\real_+),dt)$ が存在すると仮定.
これは特に
\begin{align}
    \abs{b}=
    \begin{cases}
        \text{bdd.} &\text{on}\quad [0,A]\times\real_+ \\
        0 &\text{on}\quad (A,\infty)\times\real_+
    \end{cases}
\end{align}
を満たす $A>0$ が存在するときに成り立つ.

$B$: $(\clf_t)$-BMとする.
以下のCLM
\begin{align}
    L_t=\int_0^t b(s,B_s)dB_s
\end{align}
と関連する指数martingale
\begin{align}
    D_t
    = \mathcal{E}(L)_t
    = \exp\left(\int_0^t b(s,B_s)dB_s-\frac{1}{2}\int_0^t b(s,B_s)^2 ds\right)
\end{align}
について考える.
$b$ の仮定よりThm. \ref{thm:523}の(i)が成り立つことは保証され,ゆえに $D$: UIM.
$Q:=D_\infty\cdot P$ と定める.
Girsanov's ThmとConsequencesの(c)より
\begin{align}
    \beta_t:=B_t-\int_0^t b(s,B_s)ds
\end{align}
で定まるprocessは $(\clf_t)$-BM under $Q.$

この性質はprocess $X=B$ が確率微分方程式
\begin{align}
    dX_t=d\beta_t+b(t,X_t)dt
\end{align}
の解であるような $(\clf_t)$-BM $\beta$ が確率測度 $Q$ の下で存在する,と言い換えることができる.


\end{document}
