\documentclass{jsarticle}
% \documentclass[b4paper,landscape,14pt]{jsarticle}
\title{}
\author{}
\date{
% \number\year 年 \number\month 月
}
\usepackage{fenrir_v1_4_0}
\usepackage{ethm_v1_1_0}
\mathtoolsset{showonlyrefs=true}

\begin{document}
% \maketitle
\setcounter{section}{4}
\section{Stochastic Integration}
\setcounter{subsection}{4}
\subsection{Girsanov's Theorem}

\setcounter{thm}{22}
\begin{screen}
    \begin{thm}
    \label{thm:523}
        $L$: CLM s.t. $L_0=0.$
        以下の性質を考える:
        \begin{enumerate}[label=(\roman*)]
            \item
            $E[\exp\frac{1}{2}\gen{L,L}_{\infty}]<\infty$
            (Novikov's criterion\footnote{criterion(クライテリオン):基準})
            \item
            $L$: CLMかつ $E[\exp\frac{1}{2}L_{\infty}]<\infty$
            (Kazamaki's criterion)
            \item
            $\mathcal{E}(L)$: UIM
        \end{enumerate}
        このとき,(i) $\Rightarrow$ (ii) $\Rightarrow$ (iii) の順で成立.
    \end{thm}
\end{screen}


\begin{proof}
    \begin{description}
        \item[(i) $\Rightarrow$ (ii)]
        (i) より $E[\gen{L,L}_{\infty}]<\infty$ であり,さらにThm. 4.13より $L$ は $L^2$-bdd. な連続martingale.
        このとき
        \begin{align}
            \exp\frac{1}{2}L_{\infty}
            = (\mathcal{E}(L)_{\infty})^{1/2}(\exp\frac{1}{2}\gen{L,L}_{\infty})^{1/2}
        \end{align}
        が成り立つので,Cauchy-Schwarzの不等式より
        \begin{align}
            E[\exp\frac{1}{2}L_{\infty}]
            &\le (E[\mathcal{E}(L)_{\infty}])^{1/2}
            (E[\exp\frac{1}{2}\gen{L,L}_{\infty}])^{1/2} \\
            &\le (E[\exp\frac{1}{2}\gen{L,L}_{\infty}])^{1/2}
            < \infty.
        \end{align}
        
        \item[(ii) $\Rightarrow$ (iii)]
        $L$ がUIMであることからThm. 3.22より任意のstopping time $T$ に対し $L_T=E[L_{\infty}\mid\clf_T].$
        Jensenの不等式より
        \begin{align}
            \exp\frac{1}{2}L_T
            \le E[\exp\frac{1}{2}L_{\infty}\mid\clf_T].
        \end{align}

        $E[\exp\frac{1}{2}L_{\infty}]<\infty$ という仮定より,任意のstopping time $T$ に対し $E[\exp\frac{1}{2}L_{\infty}\mid\clf_T]$ の形をしたすべてのr.v.の族は一様可積分.
        さらに先ほどの有界性より,任意のstopping time $T$ に対し $\exp\frac{1}{2}L_T$ の形をしたすべてのr.v.の族もまた一様可積分.

        $0<\Forall a<1$ に対し,$Z_t^{(a)}:=\exp(\frac{aL_t}{1+a})$ と定める.
        このとき
        \begin{align}
            \mathcal{E}(aL)_t
            = (\mathcal{E}(L)_t)^{a^2}(Z_t^{(a)})^{1-a^2}
        \end{align}
        となることがすぐにわかる.
        もし $\Gamma\in\clf$ かつ $T$: stopping timeならば,H\"{o}lderの不等式より
        \begin{align}
            E[\bm{1}_\Gamma\mathcal{E}(aL)_T]
            \le E[\mathcal{E}(L)_T]^{a^2}
            E[\bm{1}_\Gamma Z_T^{(a)}]^{1-a^2}
            \le E[\bm{1}_\Gamma Z_T^{(a)}]^{1-a^2}
            \le E[\bm{1}_\Gamma\exp\frac{1}{2}L_T]^{2a(1-a)}.
        \end{align}

        2つ目の不等号では $E[\mathcal{E}(L)_T]\le1$ という性質を用いた($\mathcal{E}(L)$: 非負supermartingaleかつ $\mathcal{E}(L)_0=1$ よりProp. 3.25が適用できる).
        3つ目の不等号では $\frac{1+a}{2a}>1$ に注意してJensenの不等式を用いた.
        任意のstopping time $T$ に対し $\exp\frac{1}{2}L_T$ の形をしたすべてのr.v.の族が一様可積分であることより,
    \end{description}
\end{proof}





\end{document}